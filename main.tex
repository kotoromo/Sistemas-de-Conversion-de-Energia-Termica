\documentclass{article}
    \usepackage[utf8]{inputenc}
    \usepackage[spanish]{babel}
    \usepackage{enumerate}
    \usepackage{siunitx}
    \usepackage{amsmath}
    \usepackage{bm}

    \begin{titlepage}
        \title{Pr\'{a}ctica 8: Ciclo Otto}
        \author{Garc\'{i}a Fierros Nicky}
        \date{26 de abril del 2018}
    \end{titlepage}
     
    \begin{document}
    \pagenumbering{gobble}
    \maketitle
    \newpage
    \pagenumbering{roman}

    \tableofcontents
    \newpage
    \pagenumbering{arabic}
    
    \section{Objetivos}
        El objetivo de la pr\'{a}ctica es determinar los par\'{a}metros de operaci\'{o}n del motor
        al realizar 2 pruebas en este: a velocidad constante y a velocidad variable.
    
    \section{Ciclo Otto}
        \subsection{Ciclos de Aire est\'{a}ndar}
            A diferencia de los ciclos de potencia de vapor, en donde se ven involucrados cambios de fase, existen otras m\'{a}quinas t\'{e}rmicas productoras de
            potencia (motores) cuyo funcionamiento no involucra un cambio de fase; esto es, que s\'{o}lo funcionan con gas. En estos motores existe un cambio
            en la composici\'{o}n del fluido de trabajo, porque durante la combusti\'{o}n cambia el fluido de trabajo de aire y combustible a una mezcla
            de productos de combusti\'{o}n. Es por esta raz\'{o}n que a estas m\'{a}quinas se les conocen como \textit{\textbf{motores de combusti\'{o}n interna}}.

            Debido a que el fluido de trabajo no pasa por un ciclo termodin\'{a}mico completo, los motores de combusti\'{o}n interna producen potencia
            utilizando \textbf{\textit{ciclos abiertos}}. Sin embargo, para analizar los motores de combusti\'{o}n interna es conveniente modelar el ciclo como uno 
            de combusti\'{o}n interna que se aproxime al funcionamiento del ciclo abierto real. Para hacer esto, es apropiado realizar las siguientes 
            consideraciones:

            \begin{enumerate}[i]
                \item Suponer una masa fija del fluido de trabajo durante todo el ciclo as\'{i} como suponer al aire como un gas ideal. De tal manera, no se
                consideran los procesos de succi\'{o}n ni explusi\'{o}n.

                \item El proceso de combusti\'{o}n es sustituido por un proceso de transferencia de calor de una fuente externa.

                \item El ciclo es completado por la transferencia de calor a su entorno, en contraste a la succi\'{o}n y explusi\'{o}n de gases que ocurre
                en el ciclo real.

                \item Todos los procesos son internamente reversibles.

                \item Se suele asumir que se trabaja con un calor espec\'{i}fico constante $\bm{Cp}$, evaluado a \SI{300}{[\kelvin]}, llamadas
                \textit{propiedades de aire fr\'{i}o}.
                
            \end{enumerate}

            El emplear estas consideraciones para el ciclo nos permite analizar de manera cualitativa a las variables que influyen en el desempe\~{n}o del ciclo.
            Es obvio concluir que, gracias a las idealizaciones que se toman en cuenta para modelar estos ciclos, los resultados cuantitativos como la 
            eficiencia t\'{e}rmica del ciclo diferir\'{a}n del ciclo real.
        
        \subsection{Motores de pist\'{o}n}
            Tambi\'{e}n conocidos como motores reciprocantes, 
        \subsection{Funcionamiento del ciclo Otto}
        
        \subsection{Eficiencia del ciclo te\'{o}rico}


    \section{Funcionamiento de inyecci\'{o}n de combustible}
    \section{Funcionamiento del carburador}
    \section{Requisitos para una buena combusti\'{o}n}
    \section{Material}
    \section{Desarrollo de la pr\'{a}ctica}
    \section{C\'{a}lculos}
    \section{Conclusiones}

    \end{document}
    
    